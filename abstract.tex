 
\begin{abstract}
Linux containers have become very popular these days due to their lightweight nature and portability. 
Numerous web services are now deployed as clusters of containers. 
Kubernetes is a popular container management system that enables users to deploy such web services easily, and hence, 
it facilitates web service migration to the other side of the world.
However, since Kubernetes relies on external load balancers provided by cloud providers, 
it is difficult to use in environments where there are no supported load balancers.
This is particularly true for on-premise data centers, or for all but the largest cloud providers.
In this paper, we proposed a portable load balancer that was usable in any environment, and hence facilitated web services migration.
We implemented a containerized software load balancer that is run by Kubernetes as a part of container cluster, 
using Linux kernel's Internet Protocol Virtual Server(IPVS).
Then we compared the performance of our proposed load balancer with existing iptables Destination Network Address 
Translation (DNAT) and the Nginx load balancers.
During our experiments, we also clarified the importance of two network conditions to derive the best performance: 
the first was the choice of the overlay network operation mode, and the second was distributing packet processing to multiple cores.
The results indicated that our proposed IPVS load balancer improved portability of web services without sacrificing the performance.
\end{abstract}
